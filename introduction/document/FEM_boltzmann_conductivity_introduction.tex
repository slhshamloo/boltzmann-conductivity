\documentclass[12pt]{article}
\usepackage{amsmath,amssymb,bbold,bm,empheq,graphicx,hyperref,siunitx,subcaption,upgreek,xcolor}
\usepackage[a4paper, margin=1in]{geometry}
\usepackage[italicdiff]{physics}
\usepackage[justification=centering]{caption}
\usepackage{subcaption}
\usepackage[section]{placeins}

\hypersetup{colorlinks=true, urlcolor=cyan}
\newcommand{\ddfrac}[2]{{\displaystyle\frac{\displaystyle #1}{\displaystyle #2}}}

\title{FEM for Boltzmann Conductivity: An Introduction}
\author{Saleh Shamloo Ahmadi}
\date{April 18, 2025}

\begin{document}
\maketitle
\section{Problem Statement}
\subsection{The Boltzmann Transport Equation}
From the Boltzmann transport equation, we have
\begin{gather}
    \dv{f}{t} = \qty(\pdv{f}{t})_{\text{coll}}, \\
    \pdv{f}{t} + \frac{\vb{p}}{m}\vdot\grad_{\vb{r}} f + \vb{F}\vdot\grad_{\vb{p}}f = I[f],
\end{gather}
where $f$ is any density function, $\vb{p}$ is the momentum of the particles, $\vb{F}$ is the
external force, and $I[f]$ is the \emph{collision integral}. In the context of conductivity of
electrons,
\begin{equation}
    \vb{F} = -e\qty(\vb{E} + \vb{v}\cross\vb{B}),
\end{equation}
and replacing the momentum $\vb{p}$ with the wavevector $\vb{k}$, we have
\begin{equation}
    \pdv{f}{t} + \frac{\hbar\vb{k}}{m}\vdot\grad_{\vb{r}} f
    - \frac{e}{\hbar}\qty(\vb{E} + \vb{v}\cross\vb{B})\vdot\grad_{\vb{k}}f = I[f].
    \label{eq:boltzmann_general}
\end{equation}
If we consider a fluctuation to the equilibrium distribution $f_0(\vb{k})$ (the Fermi-Dirac
distribution here), then,
\begin{equation}
    f(\vb{r}, \vb{k}; t) = f_0(\vb{k}) + g(\vb{r}, \vb{k}; t)
\end{equation}
(assuming the material is uniform, the equilibrium distribution does not have any $\vb{r}$
dependence). We could also rewrite the wavevector gradient by applying the chain rule and using
the relation between the group velocity and the energy. That is,
\begin{empheq}[right={\empheqrbrace\implies\grad_{\vb{k}}
    f_0=\hbar\vb{v}\displaystyle\pdv{f_0}{\varepsilon}}]{align}
    \vb{v} &= \frac{\grad_{\vb{k}}\varepsilon}{\hbar} \\
    \grad_{\vb{k}}f_0 &= \qty(\grad_{\vb{k}}\varepsilon)\qty(\pdv{f_0}{\varepsilon}),
\end{empheq}
where $\vb{v}$ is the group velocity $\varepsilon$ is the energy. Using all this, equation
\eqref{eq:boltzmann_general} becomes
\begin{equation}
    \pdv{g}{t} + \frac{\hbar\vb{k}}{m}\vdot\grad_{\vb{r}} g
    - \frac{e(\vb{E} + \vb{v}\cross\vb{B})}{\hbar}\vdot\grad_{\vb{k}}g - I[f_0 + g]
    = e\vb{E}\vdot\vb{v}\pdv{f_0}{\varepsilon} \label{eq:boltzmann}
\end{equation}
(note that $\vb{v}\cross\vb{B}$ is perpendicular to $\vb{v}$). Now, we only consider the case of
uniform fields (i.e. no $\vb{r}$ dependence), so we can drop the spatial gradient term. Also, for
the conductivity tensor, we are only interested in the linear response to the electric field
$\vb{E}$, so we can drop the electric field term on the left-hand side ($g$ itself is in first order
with respect to the electric field, so multiplying by the electric field itself gives a term of
second order in the electric field). Finally, linearizing the collision integral in terms of the
fluctuation $g$, we have
\begin{equation}
    I[f_0 + g] = \int\dd{\vb{k'}} C(\vb{k}, \vb{k'}) g(\vb{k'}; t)
    - \int\dd{\vb{k'}} C(\vb{k'}, \vb{k}) g(\vb{k}; t) + \mathcal{O}(g^2),
\end{equation} 
where $C(\vb{k}, \vb{k'})$ is a scattering kernel. I call the two terms the \emph{in-scattering} and
\emph{out-scattering} terms respectively. The out-scattering is usually expressed with the
\emph{relaxation time} $\tau$ or \emph{scattering rate} $\Gamma=1/\tau$, which is equal to
\begin{equation}
    \frac{1}{\tau(\vb{k})} = \int\dd{\vb{k'}} C(\vb{k'}, \vb{k}).
\end{equation}

\sloppy{In-scattering is dropped in the \emph{relaxation time approximation}, which is the case for
the Chambers formula. Our goal here is to keep this term, as it especially becomes important when
there is strong anisotropy in the scattering, which is the case for cuprates. When the scattering
is roughly isotropic, $C(\vb{k}, \vb{k'})$ does not strongly depend on $\vb{k}$ and, because of
charge conservation,
$\int\dd{\vb{k'}} C(\vb{k}, \vb{k'}) g(\vb{k'}; t) \approx C\int\dd{\vb{k'}} g(\vb{k'}; t) = 0.$
Due to time-reversal symmetry, $C(\vb{k}, \vb{k'}) = C(\vb{k'}, \vb{k})$, so if we consider any
anisotropy in the out-scattering, we must do the same for the in-scattering. This means the
Chambers formula is just wrong for anisotropic scattering. So, why does it work effectively
anyway? The scattering in cuprates might only be low-angle, so the in-scattering is small compared
to the out-scattering.}

Plugging in the expression for the collision integral into equation \eqref{eq:boltzmann} and
applying the approximations, we get
\begin{equation}
    \qty(\pdv{t} + \frac{1}{\tau(\vb{k})} - \frac{e(\vb{v}\cross\vb{B})}{\hbar}\vdot\grad_{\vb{k}})
    g(\vb{k}; t) - \int\dd{\vb{k'}} C(\vb{k}, \vb{k'}) g(\vb{k'}; t)
    = e\vb{E}\vdot\vb{v}\pdv{f_0}{\varepsilon}.
\end{equation}
Depending on whether we want to study steady-state DC fields or AC fields, we can set $\pdv*{t}$ to
equal to zero or perform a Fourier transform and set $\pdv*{t} = -i\omega$. For the sake of
completeness, I will go with the latter case (just setting $\omega=0$ and using the regular,
non-Fourier-trasformed fields would produce the former case easily).

This equation can be expressed as
\begin{equation}
    \int \dd{\vb{k'}} v(\vb{k}) A(\vb{k}, \vb{k'}) g(\vb{k'}) = h(\vb{k})
    \label{eq:integral_equation}
\end{equation}
where
\begin{gather}
    A(\vb{k}, \vb{k'}) = \qty(-\frac{i\omega}{v(\vb{k})} + \frac{1}{v(\vb{k})\tau(\vb{k})}
        - \frac{e\qty(\vu{v}(\vb{k})\cross\vb{B})}{\hbar}\vdot\grad_{\vb{k}})\delta(\vb{k}-\vb{k'})
        - \frac{C(\vb{k},\vb{k'})}{v(\vb{k})}, \label{eq:operator} \\
    h(\vb{k}) = e\vb{E}\vdot\vb{v}(\vb{k})\pdv{f_0}{\varepsilon}.
\end{gather}
Where $v=\norm{\vb{v}}$ and $\vu{v}=\vb{v}/v$. Separating the velocity magnitude $v(\vb{k})$
seems like an arbitrary choice here, but it is done with good reason. Later, it will make the
formula for the conductivity tensor more symmetric and makes the calculation of $A$ a bit more
streamlined. The solution can be expressed as
\begin{equation}
    g(\vb{k}) = \frac{1}{v(\vb{k})} \int\dd{\vb{k'}} A^{-1}(\vb{k}, \vb{k'}) h(\vb{k'}).
\end{equation}
So, if we could discretize everything, this would just become a linear system of equations. This is
the goal of the finite element method (FEM).

\subsection{The Conductivity Tensor}
The current density is given by
\begin{equation}
    \vb{J} = -\frac{2e}{(2\pi)^d} \int_{\text{BZ}} \dd{\vb{k}} \vb{v}(\vb{k}) f(\vb{k}),
\end{equation}
where BZ is the brillouin zone. The factor of 2 in the numerator is due to the two possible spin
configurations per wavevector $k$ and the factor $(2\pi)^d$ in the denominator comes from the
weighted sum over the wavevectors. The conductivity tensor is given by
\begin{equation}
    \sigma_{ab} = \pdv{J_a}{E_b}.
\end{equation}
Since there are no currents in equilibrium, we have
\begin{equation}
    \sigma_{ab} = -\frac{2e}{(2\pi)^d} \int_{\text{BZ}}\dd{\vb{k}} v_a(\vb{k}) \pdv{g(\vb{k})}{E_b}.
\end{equation}
plugging in the expression for $g(\vb{k})$,
\begin{equation}
    \sigma_{ab} = -\frac{2e^2}{(2\pi)^d} \int_{\text{BZ}}\dd{\vb{k}} \frac{v_a(\vb{k})}{v(\vb{k})}
    \int\dd{\vb{k'}} A^{-1}(\vb{k}, \vb{k'}) v_b(\vb{k'})\qty(\pdv{f_0}{\varepsilon}).
\end{equation}
As $T\to0$, the Fermi-Dirac distribution becomes a step function, so the term
$\pdv*{f_0}{\varepsilon}$ becomes a Dirac delta at the Fermi energy. If we apply the integral over
the delta function, an extra factor of $(\grad_{k}\varepsilon)^{-1} = (\hbar v)^{-1}$ (the Jacobian)
is added. In the end, we have
\begin{equation}
    \sigma_{ab} = -\frac{2e^2}{(2\pi)^d\hbar} \int_{\text{FS}}\dd{\vb{k}}\int\dd{\vb{k'}}
    \hat{v}_a(\vb{k}) A^{-1}(\vb{k}, \vb{k'}) \hat{v}_b(\vb{k'}),
    \label{eq:conductivity}
\end{equation}
where FS is the Fermi surface, and $\hat{v}_a=v_a/v$ is the component of the velocity in the
$a$ direction, divided by the full magnitude of the velocity.

\section{Finite Element Method}
\subsection{Discretization}
To solve this using the finite element method, we first define a set of basis functions, and then
approximate everything as a linear combination of these basis functions. So, the goal is to express
everything in equation \eqref{eq:conductivity} in terms of linear combinations of these basis
functions and transform the differential equation problem into a linear algebra problem. To do this,
we use the machinery of tensor calculus, but it is not necessary to have a deep understanding of it.
It is only used to simplify the notation.

Before we begin, remember the simple rule that when summing over an index, it should be on the top
for one quantity and on the bottom for another. If an index is not summed over, then it will
obviously stay where it is in the final quantity. We call this the \emph{index contraction} rule.
The placement of indices on the top or the bottom has a deeper connection with tensor calculus and
covariance and contravariance, but it is not needed to understand the procedures here. You can just
think of it as a syntactic game. I will also not use the Einstein summation convention, just to
make things completely explicit and hopefully clearer.

Defining the basis as the set $\{\psi_i(\vb{k})\}$, we can approximate any function by a linear
combination of these basis functions;
\begin{equation}
    f(\vb{k}) = \sum_i f^{i}\psi_i(\vb{k}). \label{eq:vector}
\end{equation}
Now, we would like to evaluate integrals over the Fermi surface. So, to make our notation simpler,
we can define quantities with subscripts to represent integrals. Also, notice these are projections
of the functions onto the basis functions.
\begin{equation}
    f_i \equiv \int\dd{\vb{k}} \psi_i(\vb{k}) f(\vb{k})
    = \sum_j f^j \int\dd{\vb{k}} \psi_i(\vb{k}) \psi_j(\vb{k})
    \label{eq:integral_vector}
\end{equation}
It is useful to define the quantity
\begin{equation}
    M_{ij} \equiv \int \dd{\vb{k}} \psi_i(\vb{k}) \psi_j(\vb{k}),
    \label{eq:overlap}
\end{equation}
which we call the \emph{overlap matrix}. Using it,
\begin{equation}
    f_i = \sum_j M_{ij} f^j.
    \label{eq:lowering_vector}
\end{equation}

We now try to define the operator in this basis. The question is, where should we put the indices
for the matrix representing the action of operator $A$ in this space? Well, the action of such
matrix on a vector $f^i$ should produce a new vector $(Af)^i$. This should be a linear combination
of the components of the vector $f^i$. So, to abide by our rule of index contraction, the symbol for
the matrix of $A$ should have one index on the bottom to sum over the elements of $f^i$ and one index
on the top for each of the elements of the resulting vector $(Af)^i$. Putting it all together,
\begin{equation}
    (Af)(\vb{k}) = \sum_i (Af)^i \psi_i(\vb{k}) \equiv \sum_{ij} A^i_j f^j \psi_i(\vb{k}).
\end{equation}
For the integrals, we have
\begin{equation}
    (Af)_i = \int\dd{\vb{k}} \psi_i(\vb{k}) (Af)(\vb{k}) = \sum_{jk} A^j_kf^k
    \int\dd{\vb{k}} \psi_i(\vb{k})\psi_j(\vb{k}) = \sum_{jk} M_{ij} A^j_k f^k.
    \label{eq:integral}
\end{equation}
To make it simpler, we can define the symbol
\begin{equation}
    A_{ij} \equiv \sum_k M_{ik} A^k_j
    \label{eq:lowering_matrix}
\end{equation}
so that
\begin{equation}
    (Af)_i = \sum_j A_{ij} f^j.
    \label{eq:integral_matrix}
\end{equation}

From equations \eqref{eq:lowering_vector} and \eqref{eq:lowering_matrix}, we can see that the
overlap matrix $M$ performs the action of \emph{lowering} the indices. If we invert this matrix
in each of these linear systems, we have
\begin{align}
    f^i &= \sum_j (M^{-1})^{ij} f_j, \\
    A^i_j &= \sum_k (M^{-1})^{ik} A_{kj}.
\end{align}
Again, the index placement is done by the index contraction rule. So we can see that the inverse of
the overlap matrix $M^{-1}$ performs the action of \emph{raising} the indices. In tensor algebra,
the equivalent of this is the metric tensor.

Finding the components of the vectors $f^i$ is easy; they are just the value of the functions at
each discrete point in our approximation space. But how do we find the elements of the matrix $A$?
The action of the operator $A$ on a function $f(\vb{k})$ is
\begin{equation}
    (Af)(\vb{k}) = \int\dd{\vb{k'}} A(\vb{k}, \vb{k'}) f(\vb{k'}).
\end{equation}
Inserting this into equation \eqref{eq:integral}, and utilizing equation \eqref{eq:integral_matrix},
we get
\begin{equation}
    \sum_j A_{ij} f^j = \int\dd{\vb{k}} \psi_i(\vb{k}) \int\dd{\vb{k'}} A(\vb{k}, \vb{k'}) f(\vb{k'}).
\end{equation}
expanding the function $f(\vb{k'})$ in terms of the basis functions from equation \eqref{eq:vector},
\begin{equation}
    \sum_j A_{ij} f^j = \int\dd{\vb{k}} \psi_i(\vb{k}) \int\dd{\vb{k'}} A(\vb{k}, \vb{k'})
    \sum_j f^j \psi_j(\vb{k'}).
\end{equation}
This is true for any function vector $f^j$, so we can drop it from both sides. This finally gives
\begin{equation}
    A_{ij} = \int\dd{\vb{k}} \int\dd{\vb{k'}} \psi_i(\vb{k}) A(\vb{k}, \vb{k'}) \psi_j(\vb{k'}).
    \label{eq:matrix_elements}
\end{equation}

Now that we know how to expand everything in this discrete basis, let us go back to our original
integral equation \eqref{eq:integral_equation}. We would like to solve
\begin{equation}
    \int \dd{\vb{k'}} A(\vb{k}, \vb{k'}) g(\vb{k'}) = (Ag)(\vb{k}) = h(\vb{k}).
\end{equation}
To express this equation in terms of the symbols we know, we can multiply both sides by
$\psi_i(\vb{k})$ and integrate over $\vb{k}$. This gives
\begin{equation}
    \int \dd{\vb{k}} \psi_i(\vb{k}) (Ag)(\vb{k}) = \int \dd{\vb{k}} \psi_i(\vb{k}) h(\vb{k}).
\end{equation}
Using equations \eqref{eq:integral_vector} and \eqref{eq:integral_matrix}, this becomes
\begin{equation}
    (Ag)_i = h_i.
\end{equation}
Expanding the left-hand side,
\begin{equation}
    \sum_j A_{ij} g^j = h_i
\end{equation}
and solving the linear system,
\begin{equation}
    g^i = \sum_j (A^{-1})^{ij} h_j.
    \label{eq:inverse}
\end{equation}
Again, the index placement is done by the index contraction rule. The important result is that
we need to invert the $A_{ij}$ form of the matrix to get the inverse we need for our case.
\begin{equation}
    (A^{-1})^{ij} = (A_{ij})^{-1}.
\end{equation}

Finally, to calculate the conductivity tensor from \eqref{eq:conductivity}, we need to multiply
$\hat{v}_a$ with $A^{-1}\hat{v}_b$ and take an integral over the result. To express this operation
in our discrete basis, consider two functions $f$ and $g$. Their product is
\begin{equation}
    f(\vb{k})g(\vb{k}) = \sum_{ij} f^i g^j \psi_i(\vb{k}) \psi_j(\vb{k}),
\end{equation}
and the integral over this product is
\begin{equation}
    \int\dd{\vb{k}} f(\vb{k}) g(\vb{k}) = \sum_{ij} f^i g^j \int\dd{\vb{k}}
    \psi_i(\vb{k})\psi_j(\vb{k}) = \sum_{ij} f^i g^j M_{ij} = \sum_i f^i g_i.
\end{equation}
So, the vector for one function needs to have indices on top and the other on the bottom. From
equation \eqref{eq:inverse}, we can see $A^{-1}\hat{v}_b$ has the form
$(A^{-1})^{ij} (\hat{v}_b)_j$, which would have an index on top after contraction. Therefore,
$\hat{v}_a$ would have to have an index on the bottom to perform the multiplication and integration.

Putting everything together, we have
\begin{equation}
    \sigma_{ab} = -\frac{2e^2}{(2\pi)^d\hbar} \sum_{ij} \qty(\hat{v}_a)_i
    \qty(A^{-1})^{ij} \qty(\hat{v}_b)_j,
\end{equation}
or, in terms of all the terms we know how to calculate,
\begin{equation}
    \sigma_{ab} = -\frac{2e^2}{(2\pi)^d\hbar} \sum_{ij} \qty[
    \qty(\sum_k M_{ik} (\hat{v}_a)^k) \qty(A_{ij})^{-1} \qty(\sum_k M_{jk} \qty(\hat{v}_b)^k)].
\end{equation}
To do the computation in practice, $M_{ij}$ and $A_{ij}$ can be calculated from \eqref{eq:overlap}
and \eqref{eq:matrix_elements} respectively, and $(\hat{v}_a)^i$ and $(\hat{v}_b)^i$ vector
components are the values of $\hat{v}_a$ and $\hat{v}_b$ at the discrete points of our approximation
space.  Afterwards, all that will be left to do is two matrix multiplications $M\hat{v}$, one linear
system solution $A^{-1}(M\hat{v}_b)$, and one dot product $(M\hat{v}_a)\vdot(A^{-1}M\hat{v}_b)$.

\subsection{Differential Operator Terms}
Splitting $A$ into the comprising terms using \eqref{eq:operator} and \eqref{eq:matrix_elements},
we have
\begin{align}
A_{ij} &= \begin{aligned}[t]
    &\int\dd{\vb{k}} \qty(-i\omega + \frac{1}{\tau(\vb{k})})
        \frac{\psi_i(\vb{k})\psi_j(\vb{k})}{v(\vb{k})} \\
    &-\int\dd{\vb{k}} \psi_i(\vb{k})\frac{e\qty(\vu{v}(\vb{k})\cross\vb{B})}{\hbar}
        \vdot\grad_{\vb{k}}\psi_j(\vb{k}) \\
    &- \int\dd{\vb{k}} \int\dd{\vb{k'}}
    \psi_i(\vb{k}) \frac{C(\vb{k},\vb{k'})}{v(\vb{k})} \psi_j(\vb{k'})
\end{aligned} \\
&\equiv \Gamma_{ij} - \frac{eB}{\hbar} \qty(D_{\vu{v}\cross\vu{B}})_{ij} - S_{ij}.
\end{align}
The matrices $\Gamma_{ij}$, $D_{\vu{v}\cross\vu{B}}$ and $S_{ij}$ are called the
\emph{out-scattering}, \emph{derivative} and \emph{in-scattering} matrices respectively.

Expanding the $k$-dependence in terms of the basis functions for the out-scattering
matrix $\Gamma_{ij}$ gives
\begin{equation}
    \Gamma_{ij} = \sum_k \int\dd{\vb{k}} \qty(-i\omega + \frac{1}{\tau_k})
    \frac{\psi_i(\vb{k})\psi_j(\vb{k})\psi_k(\vb{k})}{v_k}.
\end{equation}
Let us define the tensor $\Pi_{ijk}$ and the vector $\gamma^k$, such that
\begin{empheq}[left={\Gamma_{ij}=\sum_k\Pi_{ijk}\gamma^k\implies\empheqlbrace}]{align}
    \Pi_{ijk} &= \int\dd{\vb{k}} \psi_i(\vb{k})\psi_j(\vb{k})\psi_k(\vb{k}), \\
    \gamma^k &= -\frac{i\omega}{v_k} + \frac{1}{v_k\tau_k}.
\end{empheq}
So, for any given basis, we have to calculate the $\Pi_{ijk}$ tensor. The $\gamma^k$ vector
is given by the velocity, frequency, and scattering kernel (or possibly the scattering rate
$1/\tau$ defined ``by hand'' when ignoring in-scattering).

Expanding the $k$-dependence in terms of the basis functions for the in-scattering matrix $S_{ij}$
gives
\begin{equation}
    S_{ij} = \sum_{mn} \int\dd{\vb{k}} \int\dd{\vb{k'}} \psi_i(\vb{k})\psi_m(\vb{k})
    \frac{C_{mn}}{v_m}\psi_n(\vb{k'})\psi_j(\vb{k'}).
\end{equation}
Let us define the tensor $\Xi_{ijmn}$ and the matrix $s^{mn}$, such that
\begin{empheq}[left={S_{ij}=\sum_{mn}\Xi_{ijmn}s^{mn}\implies\empheqlbrace}]{align}
    \Xi_{ijmn} &= \int\dd{\vb{k}} \int\dd{\vb{k'}} \psi_i(\vb{k})\psi_j(\vb{k'})
    \psi_m(\vb{k})\psi_n(\vb{k'}), \\
    s^{mn} &= \frac{C_{mn}}{v_m}.
\end{empheq}
So, for any given basis, we have to calculate the $\Xi_{ijmn}$ tensor and the $s^{mn}$ matrix
is given from the scattering kernel.

\subsection{Piecewise Linear Basis for 2D Fermi Surfaces}
The 2D case is a bit simpler and can help you gain a better understanding before moving to the
full 3D case.

We choose $n$ points on $\vb{k}_i$ to represent the Fermi surface. Then, the discretization of the
Fermi surface would be the line segments $I_i$ connecting these points. These line segments are also
called the \emph{elements}. The length of each line segment is
\begin{equation}
    \ell_i = \abs{\vb{k}_{i+1} - \vb{k}_i}.
\end{equation}
Note that since our domain is a closed curve (the Fermi surface is closed), there are periodic
boundary conditions, which means the point $i=n+1$ is the same as $i=1$ and $i=0$ is the same as
$i=n$. The simplest continuous basis one can define on these elements is the piecewise linear basis.
The basis function for each point $\vb{k}_i$ on the Fermi surface is linear on the connected
elements and peaks at the point $\vb{k}_i$, equal to 1. It is zero everywhere else. So, with the
parametrization of each line segment $I_i$ as
\begin{equation}
    x: I_i \mapsto [0, 1],
\end{equation}
then the basis functions can be defined as
\begin{empheq}[left={\psi_i(\vb{k})=\empheqlbrace}]{align}
    &x,& &\vb{k} \in I_{i-1}; \\
    &1-x,& &\vb{k} \in I_i; \\
    &0,& &\vb{k} \in I_j,\ j\notin \{i-1,i\}.
\end{empheq}

We can now calculate the matrices $M_{ij}$ and $A_{ij}$ in this basis. It is easy to see that, because
each element in these matrices contains the multiplication of two basis functions, they are only
nonzero when $i$ and $j$ are associated with the same point or neighboring points.

For the overlap matrix,
\begin{equation}
\begin{aligned}
    M_{ii} &= \int\dd{k}\psi_i(\vb{k})\psi_i(\vb{k}) = \int_{I_{i-1}} \dd{\vb{k}}
    \qty[\psi_i(\vb{k})]^2 + \int_{I_i} \dd{\vb{k}} \qty[\psi_i(\vb{k})]^2\\
    &= \int_0^1 \dd{x} x^2 \ell_{i-1} + \int_0^1 \dd{x} (1-x)^2 \ell_i
    = \frac{\ell_{i-1} + \ell_i}{3},
\end{aligned}
\end{equation}
\begin{equation}
    M_{i,i+1} = M_{i+1,i} = \int_{I_i} \dd{\vb{k}} \psi_i(\vb{k})\psi_{i+1}(\vb{k})
    = \int_0^1 \dd{x} x(1-x) \ell_i = \frac{\ell_i}{6}.
\end{equation}
Therefore, the full overlap matrix is
\begin{equation}
    M_{ij} = \frac{\ell_{i-1} + \ell_i}{3}\delta_{ij}
    + \frac{\ell_i}{6}\qty(\delta_{i+1,j} + \delta_{i,j+1}).
\end{equation}

For the derivative term $D_{\vu{v}\cross\vu{B}}$, note that the part
\begin{equation}
    (\vu{v}(\vb{k})\cross\vu{B})\vdot\grad_{\vb{k}}
\end{equation}
is the directional derivative in the direction $\vu{v}(\vb{k})\cross\vu{B}$. Since the functions are
only defined on the discretized 2D rings in $k$-space, this is the dot product of the vector
$\vu{v}(\vb{k})\cross\vu{B}$ with the direction of the vector $\vb{k}_{i+1} - \vb{k}_i$ (which is
along the line segment $I_i$), which we call $\Delta\vu{k}_i$. The diagonal terms vanish, since the
integral over $I_i$ and $I_{i-1}$ are completely the same, but with opposite signs; in the
direction of $\vb{k}_{i-1}$ to $\vb{k}_i$, the function $\psi_i$ increases on $I_{i-1}$, and in the
direction of $\vb{k}_i$ to $\vb{k}_{i+1}$, it decreases on $I_i$. The neighboring terms are
\begin{equation}
\begin{aligned}
    \qty(D_{\vu{v}\cross\vu{B}})_{i,i+1} &= -\qty(D_{\vu{v}\cross\vu{B}})_{i+1,i}
        = \int_{I_i} \dd{\vb{k}} \psi_i(\vb{k})
        \frac{\qty(\vu{v}_i\cross\vu{B})\vdot\Delta\vu{k}_i}{\ell_i} \\
    &= \int_0^1 \dd{x} (1-x) \qty[\qty(\vu{v}_i\cross\vu{B})\vdot\Delta\vu{k}_i]
    = \frac{\qty(\vu{v}_i\cross\vu{B})\vdot\Delta\vu{k}_i}{2},
\end{aligned}
\end{equation}
So, the full matrix is
\begin{equation}
    \qty(D_{\vu{v}\cross\vu{B}})_{ij} = \qty(\vu{v}_i\cross\vu{B})\vdot\Delta\vu{k}_i
    \qty(\frac{\delta_{i,j+1} - \delta_{i+1,j}}{2}).
\end{equation}

For the out-scattering term $\Gamma_{ij}=\sum_k\Pi_{ijk}\gamma^k$, only the elements where $i$, $j$ and $k$ are all
associated with the same point or neighboring points are nonzero.
\begin{equation}
    \Pi_{iii} = \int_{I_i} \dd{\vb{k}} \qty[\psi_i(\vb{k})]^3 + \int_{I_{i-1}}
        \dd{\vb{k}} \qty[\psi_i(\vb{k})]^3
    = \int_0^1 \dd{x} x^3 \ell_i + \int_0^1 \dd{x} (1-x)^3 \ell_{i-1}
    = \frac{\ell_i + \ell_{i-1}}{4},
\end{equation}
\begin{equation}
    \Pi_{i+1,i,i} = \Pi_{i,i+1,i} = \Pi_{i,i,i+1} = \int_{I_i} \dd{\vb{k}}
    \qty[\psi_i(\vb{k})]^2\psi_{i+1}(\vb{k}) = \int_0^1 \dd{x} (1-x)^2 x \ell_i = \frac{\ell_i}{12},
\end{equation}
\begin{equation}
    \Pi_{i,i+1,i+1} = \Pi_{i+1,i,i+1} = \Pi_{i+1,i+1,i} = \int_{I_i} \dd{\vb{k}}
    \psi_i(\vb{k})\qty[\psi_{i+1}(\vb{k})]^2 = \int_0^1 \dd{x} x(1-x)^2 \ell_i = \frac{\ell_i}{12},
\end{equation}
\begin{equation}
    \Pi_{ijk} = \frac{\ell_{i-1} + \ell_i}{4}\delta_{ijk}
    + \frac{\ell_i}{12}\qty(\delta_{i+1,j,k} + \delta_{i,j-1,k})
    + \frac{\ell_j}{12}\qty(\delta_{i,j+1,k} + \delta_{i,j,k-1})
    + \frac{\ell_k}{12}\qty(\delta_{i,j,k+1} + \delta_{i-1,j,k})
\end{equation}
Multiplying this tensor by $\gamma^k$, we get the full $\Gamma_{ij}$ matrix.
\begin{equation}
\begin{aligned}
    \Gamma_{ij} = \qty(\frac{\ell_{i-1} + \ell_i}{4}\gamma^i + \frac{\ell_i}{12}\gamma^{i+1}
        + \frac{\ell_{i-1}}{12}\gamma^{i-1})\delta_{ij}
    &+ \frac{\ell_i}{12}\qty(\gamma^i + \gamma^{i+1})\delta_{i+1,j} \\
    &+ \frac{\ell_j}{12}\qty(\gamma^j + \gamma^{j+1})\delta_{i,j+1}
\end{aligned}
\end{equation}

Finally, for the in-scattering term $S_{ij}=\sum_{mn}\Xi_{ijmn}s^{mn}$, only the elements where
$i$, $j$, $m$, and $n$ are all associated with the same point or neighboring points are nonzero. 
\textcolor{red}{TODO: Chaitanya did not calculate this. I'll do it later.}

\subsection{Piecewise Linear Basis for 3D Fermi Surfaces}
The discretization procedure for the 3D case can be a bit more complicated. In addition to points,
we also need to define some connecting surface for sets of points. In 2D, this was easy because we
could just connect pairs of points with line segments. In 3D, to make the elements as simple as
possible, we use triangles.

We define the Fermi surface by a set of points $\vb{k}_i$ and triangles $T(i,j,k)$ connecting the
points $\vb{k}_i$, $\vb{k}_j$ and $\vb{k}_k$. Also, we define $T_i$ as the set of triangles
containing the point $\vb{k}_i$. Then, defining piecewise linear basis functions, the basis function
$\psi_i$ peaks with value 1 at the point $\vb{k}_i$, is linear on $T_i$, and is 0 everywhere else.

Each triangle is parametrized as
\begin{equation}
    (x, y): T(i,j,k) \mapsto \{(x, y);\; x\in[0, 1],\, y\in[0, 1-x] \},
\end{equation}
so that each triangle is mapped to a unit right-angled triangle.
With this mapping, the basis functions can be defined as
\begin{empheq}[left={\psi_i(\vb{k})=\empheqlbrace}]{align}
    &1 - x - y,& &\vb{k} \in T_i, (x, y)_{\max(\psi_i)} = (0, 0); \\
    &x,& &\vb{k} \in T_i, (x, y)_{\max(\psi_i)} = (1, 0); \\
    &y,& &\vb{k} \in T_i, (x, y)_{\max(\psi_i)} = (0, 1); \\
    &0,& &\vb{k} \notin T_i.
\end{empheq}
To compute integrals with this parametrization, we can just use the Jacobian of the transformation.
Here, the Jacobian is the ratio of the area of the triangle $T(i,j,k)$ to the area of the unit
right-angled triangle. Let us define it as
\begin{equation}
    \alpha(i,j,k)=\norm{(\vb{k}_j - \vb{k}_i) \times (\vb{k}_k - \vb{k}_i)}.
\end{equation}
Then, the integrals are transformed as
\begin{equation}
    \int_{T(i,j,k)} \dd{\vb{k}} = \alpha(i,j,k)\int_0^1\dd{x}\int_0^{1-x}\dd{y}
    \equiv \alpha(i,j,k)\int_T\dd{x}\dd{y}.
\end{equation}

Now, for calculating each term, note that only the elements where the indices are the same or belong
to the same triangle have nonzero overlap. \textcolor{red}{TODO: insert a figure showing this. Maybe
the same one used by Chaitanya?} Here are all the nonzero terms in the matrices:
\begin{align}
M_{ii} &= \int\dd{\vb{k}}\psi_i(\vb{k})\psi_i(\vb{k})
    = \sum_{(i,j,k) \in T_i} \int_{T(i,j,k)} \dd{\vb{k}} (\psi_i(\vb{k}))^2 \\
    &= \sum_{(i,j,k) \in T_i} \alpha(i,j,k) \int_T \dd{x} \dd{y} (1 - x - y)^2
    = \frac{1}{12} \sum_{(i,j,k) \in T_i} \alpha(i,j,k), \notag \\
M_{ij} &= \int\dd{\vb{k}}\psi_i(\vb{k})\psi_j(\vb{k})
    = \sum_{(i,j,k) \in T_i \cap T_j} \int_{T(i,j,k)} \dd{\vb{k}} \psi_i(\vb{k}) \psi_j(\vb{k}) \\
    &= \sum_{(i,j,k) \in T_i \cap T_j} \alpha(i,j,k) \int_T \dd{x} \dd{y} (1 - x - y)x
    = \frac{1}{24} \sum_{(i,j,k) \in T_i \cap T_j} \alpha(i,j,k) \notag \\
    \left(\vphantom{\sum_{(i)}}\right. &=\left.\sum_{(i,j,k) \in T_i \cap T_j}
    \alpha(i,j,k) \int_T \dd{x} \dd{y} (1 - x - y)y \right) \notag
\end{align}
\begin{align}
\Pi_{iii} &= \int\dd{\vb{k}}\psi_i(\vb{k})\psi_i(\vb{k})\psi_i(\vb{k})
    = \sum_{(i,j,k) \in T_i} \int_{T(i,j,k)} \dd{\vb{k}} (\psi_i(\vb{k}))^3 \\
    &= \sum_{(i,j,k) \in T_i} \alpha(i,j,k) \int_T \dd{x} \dd{y} (1 - x - y)^3
    = \frac{1}{20} \sum_{(i,j,k) \in T_i} \alpha(i,j,k), \notag \\
\Pi_{iij} &= \Pi_{iji} = \Pi_{jii} = \int\dd{\vb{k}}\psi_i(\vb{k})\psi_i(\vb{k})\psi_j(\vb{k})
    = \sum_{(i,j,k) \in T_i \cap T_j} \int_{T(i,j,k)} \dd{\vb{k}}
        (\psi_i(\vb{k}))^2 \psi_j(\vb{k}) \\
    &= \sum_{(i,j,k) \in T_i \cap T_j} \alpha(i,j,k) \int_T \dd{x} \dd{y} (1 - x - y)^2 x
    = \frac{1}{60} \sum_{(i,j,k) \in T_i \cap T_j} \alpha(i,j,k), \notag\\
\Pi_{ijk} &= \int_{T(i,j,k)} \dd{\vb{k}}\, \psi_i(\vb{k}) \psi_j(\vb{k}) \psi_k(\vb{k})
= \alpha(i,j,k) \int_T \dd{x} \dd{y} (1 - x - y)xy = \frac{1}{120} \alpha(i,j,k).
\end{align}

For the derivative term, first, we approximate $\hat{\vb{v}}(\vb{k})$ as a piecewise constant
unit vector field normal to each triangle. That is,
\begin{equation}
\hat{\vb{v}}(\vb{k}) = \text{sgn}(i,j,k) \vu{n}(i,j,k); \quad
\vb{n}(i,j,k) \equiv (\vb{k}_j - \vb{k}_i) \times (\vb{k}_k - \vb{k}_i),
\end{equation}
where $\hat{\vb{n}} = \vb{n}/\norm{\vb{n}}$ and $\text{sgn}(i,j,k) = \pm1$ is chosen such that
the normals always point outwards from the surface.
Also note that $\norm{\vb{n}(i,j,k)} = \alpha(i,j,k)$. The other term we need is gradient of
$\psi_j$ on triangle $T(i,j,k)$. Since $\psi_j$ is linear on this triangle and peaks at $j$,
its gradient is a constant vector pointing to $j$ and perpendicular to the opposite edge
$\vb{k}_k - \vb{k}_i$.
\begin{equation}
\grad_{\vb{k}} \psi_j(\vb{k}) = \frac{\vb{u}_j(i,j,k)}{u^2_j(i,j,k)};
\end{equation}
where $u_j(i,j,k) = \norm{\vb{u}_j(i,j,k)}$ and
\begin{equation}
\vb{u}_j(i,j,k) = (\vb{k}_j - \vb{k}_i)
- \frac{(\vb{k}_j-\vb{k}_i)\vdot(\vb{k}_k-\vb{k}_i)}{\abs{\vb{k}_k - \vb{k}_i}^2}
(\vb{k}_k - \vb{k}_i).
\end{equation}
The vector $\vb{u}_j(i,j,k)$ is obtained by projecting the vector $\vb{k}_j - \vb{k}_i$ onto the
direction perpendicular to the edge $\vb{k}_k - \vb{k}_i$. This is done by subtracting the component
in the direction of $\vb{k}_k - \vb{k}_i$ from $\vb{k}_j - \vb{k}_i$. Using these calculations, we
can finally compute the derivative term.
\begin{equation}
    \qty(\hat{\vb{v}}(\vb{k})\cross\vu{B})\vdot\grad_{\vb{k}}\psi_j
    = \qty(\grad_{\vb{k}}\psi_j\cross\hat{\vb{v}}(\vb{k}))\vdot\vu{B}
    = \text{sgn}(i,j,k)\qty(\frac{\vb{u}_j(i,j,k)\cross\vu{n}(i,j,k)}{u^2_j(i,j,k)})\vdot\vu{B}.
\end{equation}
You can simplify this expression by hand. But, using a bit of geometrical thinking, since $\vb{u}_j$
perpendicular to $\vb{k}_i - \vb{k}_k$ and $\vu{n}$ is perpendicualr to the plane of the triangle
such that it is generated by counterclockwise ("right-handed") cross products of vectors on the
plane, $\vb{u}_j\cross\vu{n}$ must be in the direction of $\vb{k}_i - \vb{k}_k$. This means
\begin{equation}
    \vb{u}_j(i,j,k)\cross\vu{n}(i,j,k) = u_j(i,j,k)
    \frac{\vb{k}_i-\vb{k}_k}{\norm{\vb{k}_i-\vb{k}_k}}.
\end{equation}
Furthermore, $u_j(i,j,k)$ and $\vb{k}_i - \vb{k}_k$ can be taken as the high and the base of
the triangle $T(i,j,k)$ respectively, so the product of their norms is twice the area of the
triangle, which is the same as the Jacobian $\alpha(i,j,k)$. Putting it all together, we have
\begin{equation}
    \qty(\hat{\vb{v}}(\vb{k})\cross\vu{B})\vdot\grad_{\vb{k}}\psi_j
    = \frac{\text{sgn}(i,j,k)}{\alpha(i,j,k)}(\vb{k}_i-\vb{k}_k)\vdot\vu{B}.
\end{equation}
Using this, we can compute the derivative matrix.
\begin{align}
    (D_{\vu{v}\cross\vu{B}})_{ij} &= \int\dd{\vb{k}}\psi_i(\vb{k})
    \qty(\vu{v}(\vb{k})\cross\vu{B})\vdot\grad_{\vb{k}}\psi_j(\vb{k}) \\
    &= \sum_{(i,j,k) \in T_i \cap T_j}\alpha(i,j,k) \int_T\dd{x}\dd{y}(1-x-y)
    \frac{\text{sgn}(i,j,k)}{\alpha(i,j,k)}(\vb{k}_i - \vb{k}_k)\vdot\vu{B} \notag \\
    &= \frac{1}{6} \sum_{(i,j,k) \in T_i \cap T_j}\text{sgn}(i,j,k)(\vb{k}_i-\vb{k}_k)\vdot\vu{B}.
    \notag
\end{align}
Note that this matrix is antisymmetric and the elements on the main diagonal are equal to zero.
The antisymmetry is apparent from the $\text{sgn}(i,j,k)$ term. The diagonal being zero is a bit
more subtle. Think about what the sum means when $i=j$. We are summing the edges not containing
the point $i$ over all triangles containing $i$. But, for any point inside a closed surface, this
just means going around the point $i$ in a closed loop. So, since we are concerned with closed
surfaces, this is always zero.

In summary, defining
\begin{empheq}[left=\empheqlbrace]{align}
    \alpha_i &= \sum_{(i,j,k) \in T_i} \alpha(i,j,k), \\
    \alpha_{ij} &= \sum_{(i,j,k) \in T_i \cap T_j} \alpha(i,j,k), \\
    \vb{d}_{ij} &= \sum_{(i,j,k) \in T_i \cap T_j} \text{sgn}(i,j,k)(\vb{k}_i - \vb{k}_k),
\end{empheq}
the different matrices can be expressed as
\begin{gather}
    M_{ij} = \frac{\alpha_i}{12}\delta_{ij} + \frac{\alpha_{ij}}{24}\delta_{\expval{ij}}, \\
    \Gamma_{ij} = \qty(\frac{\alpha_i\gamma^i}{20}
        + \sum_k\frac{\alpha_{ik}\gamma^k}{60}\delta_{\expval{ik}})\delta_{ij}
        + \qty(\frac{\alpha_{ij}}{60}(\gamma^i+\gamma^j) + \frac{\alpha(i,j,k)}{120}
            \delta_{\expval{ik}}\delta_{\expval{jk}})\delta_{\expval{ij}}\\
    (D_{\vu{v}\cross\vu{B}})_{ij} = \frac{1}{6}\vu{B}\vdot\vb{d}_{ij}\delta_{\expval{ij}}.
\end{gather}
$\delta_{\expval{ij}}$ is equal to 1 if $i$ and $j$ are neighboring points, and 0 otherwise.
Technically, due to the definitions, it is not really needed, but I include it to make things
a bit clearer.

\textcolor{red}{TODO: I need to calculate the in-scattering term for the 3D case too.}

\section{Implementation Details}
Luckily, the matrices in FEM are sparse, since there is only overlap between neighboring elements.
Even better, they are nearly \emph{banded}, meaning that the non-zero elements are only on the main
diagonal and the minor diagonals surrounding it. I say nearly, because there are also extra corner
terms which we can handle separately. This greatly speeds up the calculations and simplifies the
storage of the matrices. We store these matrices in a format called \emph{diagonal ordered form}.

First, let us consider the 2D case which is simpler. Since every element only interacts with its two
neighbors, only the main diagonal and the two minor diagonals next to it are nonzero, with two extra
elements in the corner for the interaction of the last element with the first element (since the
curve is closed and there are periodic boundary conditions). This gives, for example, some $n+1$ by
$n+1$ matrix
\begin{equation}
    \mqty(b_0 & a_0 & 0 & 0 & 0 & \dots & c_0 \\
          c_1 & b_1 & a_1 & 0 & 0 & \dots & 0 \\
          0 & c_2 & b_2 & a_2 & 0 & \dots & 0 \\
          0 & 0 & c_3 & b_3 & a_3 & \dots & 0 \\
          \vdots & \vdots & \vdots & \ddots & \ddots & \ddots & \vdots \\
          0 & 0 & 0 & \dots & c_{n-1} & b_{n-1} & a_{n-1} \\
          a_n & 0 & 0 & \dots & 0 & c_n & b_n \\)
\end{equation}
which can be stored as
\begin{equation}
    \mqty(a_n & a_0 & \dots & a_{n-2} & a_{n-1} \\
        b_0 & b_1 & \dots & b_{n-1} & b_n \\
        c_1 & c_2 & \dots & c_n & c_0).
\end{equation}
The first and last rows are not typos; this is usually used for purely banded matrices, without the
extra corner terms, and the convention is made to make sense for that ($a_n$ and $c_0$ are just
missing in that case).

There is an annoying little detail with our matrices: since we are dealing with closed surfaces,
we have periodic boundary conditions. So, in addition to the diagonals, some elements near the
corners of the matrices are also nonzero. This means our matrices are not purely banded. This
does not pose a problem for storing them (we just use the corner terms usually unused by others),
but it means the inversion algorithms made for banded matrices will not work out of the box. To fix
this, we can utilize the Sherman--Morrison--Woodbury formula. First, let us consider the simpler 2D
case, where there are only two extra nonzero elements exactly in the corners. So, our matrices are
the sum of a banded matrix and a rank-1 update. The simpler Sherman--Morrison formula states that,
given some matrix $B$ (here, the banded portion) and a rank-1 update $uv^T$, the inverse of the full
matrix is
\begin{equation}
    A^{-1} = (B + uv^T)^{-1} = B^{-1} - \frac{B^{-1}uv^TB^{-1}}{1 + v^TB^{-1}u}.
\end{equation}
In our case, we can choose the matrices such that
\begin{equation}
    B = \mqty(b_0 - \gamma & a_0 & 0 & \dots & 0 \\
          c_1 & b_1 & a_1 & \dots & 0 \\
          \vdots & \ddots & \ddots & \ddots & \vdots \\
          0 & \dots & c_{n-1} & b_{n-1} & a_{n-1} \\
          0 & \dots & 0 & c_n & b_n - \flatfrac{c_0a_n}{\gamma}),\ 
    u = \mqty(1 \\ 0 \\ \vdots \\ 0 \\ a_n/\gamma),\ 
    v = \mqty(\gamma \\ 0 \\ \vdots \\ 0 \\ c_0).
\end{equation}

\section{Results}
\subsection{Free Electrons}

Figure \ref{fig:test} compares the results from my FEM implementation to the Drude model for free
electrons. Figure \ref{fig:error} shows the maximum error in every magnetic field as a function of
the number of elements. The error reduces quadratically with the number of elements
($\mathrm{Error}\sim n^{-2}$). Table \ref{tab:performance} compares the runtime of my implementation
with and without the bandedness optimization. Figure \ref{fig:performance} shows the performance
of the banded implementation as a function of the number of elements.

\begin{figure}
    \centering
    \includegraphics[width=0.8\textwidth]{figures/free_electron_test}
    \caption{Boltzmann FEM model with 100 elements vs the Drude model for free electrons.}
    \label{fig:test}
\end{figure}

\begin{figure}
    \centering
    \includegraphics[width=0.8\textwidth]{figures/free_electron_error}
    \caption{Maximum error in the conductivity tensor in different fields as a function of the
        number of elements.}
    \label{fig:error}
\end{figure}

\begin{figure}
    \centering
    \includegraphics[width=0.8\textwidth]{figures/free_electron_performance}
    \caption{Runtime of the banded implementation for 10 different magnetic fields as a function
        of the number of elements.}
    \label{fig:performance}
\end{figure}

\begin{table}
    \centering
    \begin{tabular}{|c|c|c|}
        \hline
        \# Elements & Non-banded & Banded \\
        \hline
        10 & $\SI{123}{\milli\second}$ & $\SI{40.7}{\milli\second}$ \\
        100 & $\SI{154}{\milli\second}$ & $\SI{44.8}{\milli\second}$ \\
        200 & $\SI{184}{\milli\second}$ & $\SI{51.2}{\milli\second}$ \\
        500 & $\SI{273}{\milli\second}$ & $\SI{65.4}{\milli\second}$ \\
        1000 & $\SI{663}{\milli\second}$ & $\SI{87.1}{\milli\second}$ \\
        \hline
    \end{tabular}
    \caption{Runtime of the different implementations for the free electron model.
        The calculation is done for 1000 different magnetic fields.}
    \label{tab:performance}
\end{table}

\subsection{Cuprates}
\textcolor{red}{TODO: After I complete the code for the quasi-2D case, we can compare it to the
    relaxation time approximation package.}
\end{document}
